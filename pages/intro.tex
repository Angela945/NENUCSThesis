\chapter{第一章 绪论}
\label{cha:intro}
\section{什么是 \LaTeX }
在说 \LaTeX \cite{companion}之前,必须先说 \TeX \cite{tex}。话说那是在计算机领域的神话时代,大神Knuth,
就是写《计算机程序设计的艺术》那位,40多就功成退休了。 Knuth大神自己写书,看
了出版商给的样稿,心想你们这出版商这破排版也太烂了吧,把老子帅气的书都排成便
便了。所以大神就宅在家里几天捣鼓了一个专门用于科学论文和书籍的排版软件,这套
软件就是\TeX。

\TeX 非常强大,可以排版任何样式的文稿 \pozhehao 只要你掌握了\TeX 的那近千条
原始命令。由于这个条件有一些偏激,因此在后来的几年, 利用\TeX 的宏功能开发的
宏库 \LaTeX 产生了。

关于\LaTeX 和 \TeX 的关系,可以认为 \TeX 是 \LaTeX 的基石,各类扩展宏包则是
\LaTeX 宏伟大厦上优雅的装饰。

插播八卦:\TeX 的版本号很有意思,即无限接近圆周率 $\Pi$,每修订一次增加一位
。目前我使用的TeXLive套件中,版本号已经到了3.1415826。按照Knuth大神自己的说
法,等其百年后,Tex的所有bug当作feature就好了,此等大神风范也大概只有 Knuth
能做出来(还记得他的“256 pennies for a bug”计划么?)。

\section{为什么用 \LaTeX }
\subsection{\LaTeX 公式,天下最好,没有之一}
\label{subsec:best-formula}
由于基于 \TeX ,\LaTeX 系统是公认的数学公式排得最好的系统。用LaTeX排版的公式
能够跟正文风格保持一致,而且漂亮。 Word排的公式,只能用“丑陋”形容了,如
inline mode的公式往往把正文的行距推的很宽。而且Word公式编辑器要用鼠标不停的
点啊点的,效率低下。

美国数学学会(AMS)鼓励数学家们使用 TeX 系统向它的期刊投稿,因此大量的国际会
议也都要求使用使用 TeX写论文\cite{acm-proceedings-templates}。世界上许多一流
的出版社如 Kluwer、Addison-Wesley、牛津大学出版社等也利用 TeX 系统出版书籍和
期刊。

\subsection{设备无关,关注内容} 
\LaTeX 系统的排版结果 DVI(DeVice Independent)文件与输出设备无关。DVI 文件
可以显示、打印、照排,几乎可以在所有的输出设备上输出。TeX 排版源文件及结果在
各种计算机系统上互相兼容。

这让文稿编写者在写作过程中减少在格式上的纠结,更多地去关注内容本身。想想每次
在Word上调整图片后为了保持后续页面的整洁需要折腾多久?

至于插入的图表啦,公式啦,参考文献啦,\LaTeX 都可以完成自动编号。

所以,写的时候呢,不用管排版,主要打字就可以了~。排版的时候呢,不想伤脑筋的
话就直接套用别人的模板就好 \pozhehao 比如你现在看到的:)

\section{最后}
我并不是说M\$ Word是一个一无是处的软件,只是在科技论文排版方面,\LaTeX 对
Word等工具来说就是一个无法超越的存在。虽然使用 \LaTeX 需要一定的学习曲线,但
是我保证这些时间物超所值。想把你的论文,或者书稿排的很完美,就用 \LaTeX 吧。

恩,反正这一章就是绪论。

